\documentclass[a4paper,titlepage,12pt]{article}
\usepackage[utf8]{inputenc}
\usepackage[T1]{fontenc}
\usepackage[margin=1in]{geometry}
\usepackage{parskip}
\usepackage{hyperref}
\usepackage{titlesec}
\hypersetup{
	colorlinks,
	pdfauthor=Delan Azabani,
	pdftitle=Software Engineering 200: Mars rover assignment
}

\title{Software Engineering 200:\\Mars rover assignment}
\date{October 21, 2014}
\author{Delan Azabani}

\pagenumbering{gobble}
\thispdfpagelabel0

\titlespacing*\section{0pt}{0.5em}{0.5em}

\begin{document}

\maketitle
\pagenumbering{arabic}

To compile and run this assignment submission, use the following
commands:

\texttt{\% ant \&\& java -jar dist/rover.jar}

\section{Part (c): system walkthrough}

Let's say that mission control sent the following message to our rover
on Mars:

\texttt{analyse;translate:-100;photograph;rotate:-180}

Each message represents a task list, where the tasks are separated by
semicolons. Each task has a name, such as one shown above or
\texttt{call}, which executes a previous task list. Parameters for a
task such as angles, distances and task list numbers are written after
their task name, separated by a colon.

From the perspective of our rover's system, this message starts at
\texttt{Comm.receive(String)}, which is implemented by
\texttt{ConcreteComm.receive(String)}. Because \texttt{ConcreteComm}
is a subject in the Observer pattern, all \texttt{receive()} needs to
do is call the subject's method named \texttt{notifyObservers()}, which
required by the \texttt{ObservableDevice} interface.

\texttt{notifyObservers()} then iterates through each of the
\texttt{DeviceObserver} objects who are listening, calling their
\texttt{update()} methods with the message string from
\texttt{receive()}. In practice, the only observer of
\texttt{ConcreteComm} is one \texttt{CommObserver} object.

\texttt{CommObserver.update()} takes the message and simply downcasts
it to a \texttt{String}. Then it finishes by passing it to
\texttt{controller.messageReceived()}, where \texttt{controller} is the
\texttt{RoverController} that owns the \texttt{CommObserver}.

\texttt{RoverController.messageReceived()} instantiates a
\texttt{RoverCommandCodec} to parse the message and turn it into a task
list. There's an unfortunate level of coupling here because the task
objects that its \texttt{decode()} method creates need to have
references to various devices (a \texttt{Camera}, a \texttt{Driver} and
a \texttt{SoilAnalyser}). Worse still, a \texttt{RecursionCommand}
needs to be able to read existing task lists and modify the execution
stack. As a result, the \texttt{RoverController} needs to pass many
parameters to \texttt{decode()}.

In return, we now have a \texttt{RoverTaskList} containing a
\texttt{RoverCommand} for each task in the incoming message, and we
append this task list to \texttt{program}, a \texttt{RoverListList}
which represents a list of task lists. What a mouthful.

In another thread, \texttt{RoverController.start()} has been polling
an execution stack called \texttt{stack} jadedly, because thus far it
has remained empty. The stack simply contains \texttt{Iterator} objects
inside active task lists. As there's now a task list in
\texttt{program}, we take an iterator for it and push it onto the
stack. Whenever \texttt{busy} is \texttt{false} (iff no tasks are
running) and there's a task list on the stack, we grab the next task
with \texttt{Iterator.next()} and call its \texttt{execute()} method.

\texttt{AnalysisCommand.execute()} calls
\texttt{soilAnalyser.analyse()} using its private reference field, and
\texttt{ConcreteSoilAnalyser} sends the results of the analysis back to
the controller using the same Observer workflow discussed previously.

Finally, \texttt{RoverController.analysisReceived()} calls
\texttt{Comm.send()} and Earth is pleased with the rover. Rinse and
repeat with the rest of the tasks and task lists.

\section{Part (d): design patterns}

The Observer pattern is essential and works well with the asynchronous
nature of this system. It's significantly cleaner than setting and
polling for changes in global state.
\texttt{Concrete\{Comm,Driver,SoilAnalyser,Camera\}} are the subjects,
each implementing the \texttt{ObservableDevice} interface, while
\texttt{\{Comm,Driver,SoilAnalyser,Camera\}Observer} are the observers,
implementing \texttt{DeviceObserver}. \texttt{RoverController} can even
be seen as an overarching observer of the other observers.

Dependency injection is used to decouple the \texttt{RoverController}
from any particular set of concrete device classes that it uses.
\texttt{RoverProgram} is the entry point of execution, and it is
responsible for injecting instances of
\texttt{Concrete\{Comm,Driver,SoilAnalyser,Camera\}} or any subclasses
of these into a new controller object.

The Factory pattern is realised through \texttt{RoverCommandCodec}. By
refactoring task list message parsing out of \texttt{RoverController},
the code which instantiates various subclasses of \texttt{RoverCommand}
is isolated in its own class without any other functionality.

\texttt{RoverCommand} is an interface that demonstrates the Command
pattern in combination with
\texttt{\{Rotation,Translation,Analysis,Photography,Recursion\}Command},
the classes that implement it. The key benefit of using this pattern
here is that it allows the code in \texttt{RoverController} to deal
purely with coordinating the execution stack, while treating tasks as
opaque entities. All the controller does is choose the time of
execution.

The container classes \texttt{RoverListList} and \texttt{RoverTaskList}
are subjects of the Iterator pattern, as it allows the
\texttt{RoverController} to keep track of a `cursor' in each task list
without the need for clumsy integral indices. Actually I ended up being
unable to take advantage of the pattern with \texttt{RoverListList},
because the iterator would die whenever the list of task lists is
modified, which occurs whenever a new task list arrives.

\section{Part (e): alternative design choices}

With my final design, I couldn't take advantage of Java's native
implementation of the Observer pattern, because \texttt{Observable}
is a class, not an interface, and I was already using my implementation
inheritance relationship with \texttt{Comm}, \texttt{Driver} and the
like. I could have instead decorated \texttt{Concrete*} with four
classes that extended \texttt{Observable}. While this would obviate the
need for boring Observer code, it would introduce delegate methods
instead. I chose the route with fewer classes overall.

\texttt{RecursionCommand} is unlike the other task classes, because it
doesn't really \textit{do} anything other than push an old task list
reference onto the execution stack. There's no device associated with
it, and thus no observer to tell the controller to reset \texttt{busy}.
I ended up working around this by `tagging' the other task classes with
the empty interface \texttt{AsynchronousCommand}, and checking for this
`tag' with \texttt{instanceof} after firing off \texttt{execute()}. One
could argue that this may be an anti-pattern, as it's a bit of a
semantic abuse of the interface language feature. After finishing the
assignment, I discovered that this technique was common before Java
introduced annotations; had I known about annotations earlier, I would
have certainly preferred to use them instead.

\section{Part (f): separation of concerns}

At the risk of rehashing my answer to part (d), virtually every design
pattern I used was chosen at least partially due to its capacity to
improve the separation of concerns within classes inside the Mars rover
system.

The Observer pattern separated what was to happen after a device had
new information, from each device's specific events and their
implementations.

The Factory pattern separated the choice of concrete task classes from
\texttt{RoverController}, while dependency injection separated the
choice of device classes from its constructor.

The Command pattern separates the `when' from the `how' of executing
each task. 

The Iterator pattern separates the act of walking through tasks from
the underlying data structure used, because only the former is relevant
to the controller class.

\section{Part (g): avoiding unnecessary duplication}

The design used in this assignment does not use any delegate methods or
decoration classes outside of \texttt{RoverListList} and
\texttt{RoverTaskList}, which are basically encapsulating
\textit{semantic} sugar around what are essentially \texttt{ArrayList}
objects.

I've taken advantage of the existing relevant exception class
\texttt{IllegalArgumentException} rather than creating my own subclass
just for this system. For readers of the source code, this provides the
slight additional benefit of familiarity.

The use of tedious \texttt{switch}-like constructs on external input
has been restricted to one such usage inside
\texttt{RoverCommandCodec.decode()} only.

The frequently used \texttt{Thread.sleep} wrapped in a
\texttt{try}-\texttt{catch} for \texttt{InterruptedException} has
been refactored into a static method in \texttt{RoverUtils} for
clarity of reading.

\section{Part (h): testability}

As the name suggests, the package \texttt{com.azabani.java.rover.fake}
contains fake versions of the classes \texttt{Comm}, \texttt{Driver},
\texttt{SoilAnalyser} and \texttt{Camera}. While these already
existed hypothetically, and are external to the system, they were
important for four reasons:

\begin{itemize}
	\item They allowed me to build the system during development
	      and check the syntax;
	\item The compiler would also check data types, a weak form of
	      semantic verification;
	\item I could craft a limited set of test data to use as an
	      occasional sanity check; and
	\item Being able to run parts of the system while writing it
	      allowed me to better visualise where I was going with the
	      system, because I felt less `blind'.
\end{itemize}

Given more time, I would have developed unit test harnesses using JUnit
and Mockito to test the system in a much more rigorous manner,
employing the techniques covered in \textit{Software Engineering 110}.
The use of the Factory pattern and dependency injection would certainly
help with unit testing, because of the way they effectively decouple
classes.

\end{document}
